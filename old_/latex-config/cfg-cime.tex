\usepackage[utf8]{inputenc}
\usepackage[spanish,es-tabla]{babel}
\decimalpoint
\let\cleardoublepage\clearpage
\usepackage[bitstream-charter]{mathdesign}
\usepackage[T1]{fontenc}
\newcommand{\selectSans}{\usefont{T1}{qhv}{m}{n}\selectfont} % sans-serif TeX Gyre Heros font
\usepackage{parskip}
\usepackage{amsmath}
\usepackage{amsfonts}
\usepackage{amssymb}
\usepackage{mathtools}
\usepackage{color}
\usepackage{graphicx}
\usepackage{makeidx}
\usepackage{pdflscape}
\makeindex
\usepackage{anysize}
\usepackage{anyfontsize}
\usepackage{pdfpages}
\usepackage[x11names,table]{xcolor}
\usepackage{tikz}
\usepackage{tcolorbox}
\tcbuselibrary{skins,breakable,listings,theorems}
\usepackage[hidelinks]{hyperref}
\usepackage[labelfont=bf]{caption}
\usepackage{subcaption}
\usepackage{float}
\captionsetup[table]{labelsep=space}
\captionsetup[figure]{labelsep=space}
\usepackage{setspace} % Para modificar espacios/interlineados
\usepackage{listings}
\usepackage{array,ragged2e}
\usepackage{multirow}
\usepackage{multicol}
\usepackage{enumitem}
% \usepackage{structuralanalysis}
\usepackage[titletoc]{appendix} % Para apéndices
\usepackage[left=2cm,top=2cm,right=2cm,bottom=2cm]{geometry}
\setlength{\parindent}{0cm}
\usepackage[printwatermark]{xwatermark}
\newwatermark[allpages,color=gray!10,angle=45,scale=3,xpos=0,ypos=0]{Borrador}
\tcbset{colback=green!5!white, colframe=gray!10!black, coltitle=green!20!black, 
fonttitle=\bfseries, colbacktitle=white, coltext=gray!30!black}
\addto\captionsspanish{
  \renewcommand{\figurename}{{\bf Figura}}% 
}
\addto\captionsspanish{
  \renewcommand{\chaptername}{{\bf}}% 
}
\usepackage{cancel}
\usepackage{textcomp}
\usepackage{gensymb}
\usepackage{newunicodechar}
\newunicodechar{°}{\degree}
\renewcommand{\vec}[1]{\boldsymbol{\mathbf{#1}}}
\renewcommand{\vecg}[1]{\boldsymbol{#1}}

\newcommand{\colvec}[3]{\begin{bmatrix}#1\\#2\\#3\end{bmatrix}}

\usepackage{epigraph}
\usepackage{fontawesome}
\usepackage[Bjornstrup]{fncychap}

\renewcommand{\baselinestretch}{1.00}
\newcommand{\hl}[1]{%
  \colorbox{red!50}{$\displaystyle#1$}} % Para cuadros colorizados

\newcommand{\answer}[1]{%
  \colorbox{green!50}{$\displaystyle#1$}} % Para cuadros colorizados
\newcommand{\sifu}[3]{\langle #1 - #2 \rangle^{#3}} % Funciones de singularidad

% \renewcommand{\familydefault}{\sfdefault}
\renewcommand{\vec}[1]{\mathbf{#1}}
\newcommand{\iszero}[1]{\cancelto{0}{#1}}




% Colores
\definecolor{verdep}{RGB}{166,206,58}
\definecolor{ccap}{RGB}{10,10,50}
\definecolor{csec}{RGB}{50,50,100}
\definecolor{csubsec}{RGB}{80,80,120}
\definecolor{header_table_color}{RGB}{200,255,180}
\definecolor{info_color}{RGB}{100,100,200}
\definecolor{csol}{rgb}{0.2,0.8,0.1}
\definecolor{backcode}{rgb}{0.98,0.98,0.99}
\definecolor{crule}{rgb}{0.9,0.9,0.9}
\definecolor{dkgreen}{rgb}{0,0.6,0}
\definecolor{gray}{rgb}{0.5,0.5,0.5}
\definecolor{mauve}{rgb}{0.58,0,0.82}
\definecolor{probsec}{RGB}{3,165,183}
\definecolor{codeinline}{RGB}{30,30,150}


% \newtcolorbox{ejemplo}[2][]
% {
%   breakable,
%   colframe = gray!100,
%   colback  = gray!0,
%   coltitle = gray!20!black,
%   title    =  \faEdit \hspace{5 mm} #2,
% }

\newcommand{\uno}{ ( \faStarO ) }
\newcommand{\dos}{ ( \faStarO\faStarO ) }
\newcommand{\tres}{ ( \faStarO\faStarO\faStarO ) }


\definecolor{extitle}{RGB}{0,50,0}
\definecolor{extitleback}{RGB}{225,225,225}
\definecolor{exback}{RGB}{255,255,255}
\definecolor{probtitleback}{RGB}{200,200,100}
\definecolor{probback}{RGB}{255,255,255}

\definecolor{solucolor}{RGB}{150,50,50}
\newcommand{\solu}{{\it\color{solucolor} Solución}}


\newtcolorbox[auto counter,number within=section]{ejemplo}[2][]{
  breakable,
  enhanced,
  colback=exback,
  boxrule=0.75pt,
  arc=6pt,
  % outer arc=0pt,
  title=\faEdit \hspace{5 mm}Ejemplo~\thetcbcounter \hspace{3mm} #2,
  fonttitle=\bfseries\sffamily\normalsize\strut,
  coltitle=extitle,
  colbacktitle=extitleback,
  % title style={exercisebgblue},
}

\newtcolorbox[auto counter,number within=section]{problema}[2][]{
  breakable,
  enhanced,
  colback=probback,
  boxrule=0.75pt,
  arc=6pt,
  % outer arc=0pt,
  title=\faEdit \hspace{5 mm}Problema~\thetcbcounter \hspace{3mm} #2,
  fonttitle=\bfseries\sffamily\normalsize\strut,
  coltitle=extitle,
  colbacktitle=extitleback,
  % title style={exercisebgblue},
}

\definecolor{infotitle}{RGB}{0,0,50}
\definecolor{infotitleback}{RGB}{195,240,255}
\definecolor{infoback}{RGB}{242,252,255}
\definecolor{infoframe}{RGB}{50,50,50}

\newtcolorbox{informacion}[2][]
{
  breakable,
  colframe = infoframe,
  colback  = infoback,
  coltitle = infotitle,
  colbacktitle = infotitleback,
  title    = \faInfo \hspace{5 mm} #2,
}

\newtcolorbox{recomendacion}[2][]
{
  breakable,
  colframe = green!25,
  colback  = green!10,
  coltitle = green!20!black,
  title    = #2,
}


% \newtcolorbox{outscript}[1][]
% {
%   breakable,
%   colframe = green!25,
%   colback  = green!10,
%   coltitle = green!20!black,
% }


\newcolumntype{P}[1]{>{\centering\arraybackslash}m{#1}}
\newcommand{\ccol}{>{\centering\tt\arraybackslash}}

% Nuevos comandos

\usepackage{titlesec}%--
% \newcommand{\hsp}{\hspace{5pt}}
% \titleformat{\chapter}[hang]{\huge\bfseries\color{ccap}}
% {\color{verdep}{\vrule height 2.5cm width 1mm}\hsp{\fontsize{100}{5}\selectfont\thechapter}\hsp%
% {\vrule height 2.5cm width 1mm}\hsp{\fontsize{30}{5}\selectfont}}{5pt}{\huge\bfseries}

\titleformat{\section}[hang]{\normalfont\color{csec}}%
{\filright\large\enspace\thesection\enspace}%
{8pt}{\Large\bfseries\filright}%

\titleformat{\subsection}[hang]{\normalfont\color{csec}}%
{\filright\large\enspace\thesubsection\enspace}%
{8pt}{\large\bfseries\filright}%


% For programming 
\newcommand{\code}[1]{{\tt\bfseries\color{codeinline} #1}}
\newcommand{\link}[1]{{\sffamily\color{codeinline}\url{#1}}}


% Code
\lstnewenvironment{python}{\lstset{frame=single,
  frameround=tttt,
  backgroundcolor=\color{backcode},
  rulecolor=\color{crule},
  language=python,
  aboveskip=4mm,
  belowskip=4mm,
  showstringspaces=false,
  columns=flexible,
  basicstyle={\small\ttfamily},
  numbers=none,
  numberstyle=\tiny\color{gray},
  keywordstyle=\color{blue},
  commentstyle=\color{dkgreen},
  stringstyle=\color{mauve},
  breaklines=true,
  breakatwhitespace=true,
  tabsize=4,
  extendedchars=true,
  inputencoding=utf8,
  literate=%
  {°}{{\,\,$^\circ$\,\,}}1
  {á}{{\'a}}1
  {é}{{\'e}}1
  {í}{{\'i}}1
  {ó}{{\'o}}1
  {ú}{{\'u}}1
  {Á}{{\'A}}1
  {É}{{\'E}}1
  {Í}{{\'I}}1
  {Ó}{{\'O}}1
  {Ú}{{\'U}}1
  {α}{{$alpha$}}1
  {₁}{{$ _1 $}}1
}}{}



\definecolor{backout}{RGB}{220,220,230}
\definecolor{textout}{RGB}{10,10,110}

% Code
\lstnewenvironment{outscript}{\lstset{frame=single,
  frameround=tttt,
  backgroundcolor=\color{backout},
  rulecolor=\color{crule},
  language=python,
  aboveskip=0mm,
  belowskip=4mm,
  showstringspaces=false,
  columns=flexible,
  basicstyle={\small\ttfamily\color{textout}},
  numbers=none,
  numberstyle=\tiny\color{gray},
  keywordstyle=\color{blue},
  commentstyle=\color{dkgreen},
  stringstyle=\color{mauve},
  breaklines=true,
  breakatwhitespace=true,
  tabsize=4,
  extendedchars=true,
  inputencoding=utf8,
  literate=%
  {°}{{\,\,$^\circ$\,\,}}1
  {á}{{\'a}}1
  {é}{{\'e}}1
  {í}{{\'i}}1
  {ó}{{\'o}}1
  {ú}{{\'u}}1
  {Á}{{\'A}}1
  {É}{{\'E}}1
  {Í}{{\'I}}1
  {Ó}{{\'O}}1
  {Ú}{{\'U}}1
  {α}{{$alpha$}}1
  {₁}{{$ _1 $}}1
}}{}

